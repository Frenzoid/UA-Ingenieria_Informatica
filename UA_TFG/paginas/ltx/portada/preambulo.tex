\thispagestyle{empty}
{\Huge{ \textbf{ Preámbulo }}}

\begin{center}
\line(1,0){\textwidth}
\end{center}

Hoy en día existen muchas plataformas de crowdfunding, plataformas y organizaciones orientadas a la recaudación de fondos para la subvención de proyectos. Durante los últimos años estas plataformas han conseguido una gran popularidad debido a que gracias a estas plataformas se han podido poner en marcha tanto proyectos como pequeñas startups, pero también ha atraído la atención de otros actores fraudulentos, provocando así un surgimiento cada vez mayor de estafas.

\bigskip

Las estafas en el crowdfunding abarca muchos campos, y se dan cuando una campaña de crowdfunding solicita y acepta donaciones a través de falsos pretextos, engañando a las personas sobre el resultado, la naturaleza del proyecto y/o la causa solicitada. También se dan casos de malversación de fondos, o el uso del proyecto para el blanqueo de capital, ya sea modificando el valor de las donaciones recibidas, o mediante una idea ilusoria de responsabilidad ecológica ( greenwashing ) \cite{1}.

\bigskip

Para solventar estos problemas, el objetivo de este trabajo y solución se que propone en este documento es el uso de la tecnología blockchain para proveer de trazabilidad, confianza y inmutabilidad a los activos y eventos que se realizan durante el proceso de desarrollo de dichos proyectos en una plataforma de crowdfunding, de tal forma que los fondos donados se desbloquearán para el proyecto a desarrollar a medida que se alcanzan ciertos hitos además de trazar el uso de dichos fondos.

\bigskip

Para ello, en este trabajo se estudiará las tecnologías necesarias para desarrollar una plataforma web de crowdfunding y la integración de esta con una blockchain como las tecnologías necesarias para desplegar y desarrollar un sistema capaz de gestionar un activo en la blockchain.