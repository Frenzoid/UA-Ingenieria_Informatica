\documentclass{article}
\usepackage[a4paper, total={6in, 8in}]{geometry}
\usepackage{amsmath,amsthm,amssymb,amsfonts, fancyhdr, color, comment, graphicx, environ}
\usepackage{xcolor}
\usepackage{mdframed}
\usepackage[shortlabels]{enumitem}
\usepackage{indentfirst}
\usepackage{hyperref}
\usepackage{wrapfig}
\usepackage{lipsum}
\usepackage{comment}
\usepackage{caption}
\usepackage{subcaption}
\usepackage[T1]{fontenc}
\usepackage{listings}
\usepackage{minted}
\usepackage{tcolorbox}
\usepackage{pdfpages}
\usepackage[backend=biber,style=numeric]{biblatex}
% numeric is about equal to Bibtex's plain
\usepackage[spanish]{babel}  
%LIBRERIA PARA IMAGENES NO SEAN PUTISIMO RANDOM !!!
\usepackage{graphicx}
\usepackage{listings}
\usepackage{xcolor}
\usepackage{listings}
\usepackage{minted}
\usepackage{emptypage}
\tcbuselibrary{listings, minted, skins}
\renewcommand{\footrulewidth}{0.8pt}
\definecolor{bg}{rgb}{0.9, 0.9, 0.9}
\addbibresource{config/refs.bib} % The filename of the bibliography
\usepackage[autostyle=true]{csquotes} % Required to generate language-dependent quotes in the bibliography

% Titulo de cada índice
\addto{\captionsspanish}{\renewcommand{\contentsname{Índice de Contenidos}}}
\addto{\captionsspanish}{\renewcommand{\listfigurename}{Índice de Figuras}}
\renewcommand\listoflistingscaption{Índice de Códigos}

% Cabeceras y Pies de Pagina.
\fancyhead[L]{ \begin{picture}(0,0) \put(0,0){\includegraphics[width=10mm]{img/logotipos/eps.png}} \end{picture} \hspace{0.8cm} Escuela Politécnica Superior}
\fancyhead[R]{Universidad de Alicante}

\fancyfoot[L]{Ingeniería Informática}
\fancyfoot[C]{Trabajo Fin de Grado}
\fancyfoot[R]{\thepage}

\pagestyle{fancy}