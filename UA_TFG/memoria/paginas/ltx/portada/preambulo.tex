\thispagestyle{empty}
{\Huge{ \textbf{ Preámbulo }}}

\begin{center}
\line(1,0){\textwidth}
\end{center}

Hoy en día existen muchas plataformas y organizaciones orientadas a la gestión de la subvención de proyectos, como por ejemplo las plataformas de crowdfunding. Durante los últimos años estas plataformas han conseguido una gran popularidad debido a que gracias a estas plataformas se han podido poner en marcha tanto proyectos como pequeñas startups, pero también ha atraído la atención de otros actores fraudulentos, provocando así un surgimiento cada vez mayor de estafas.

\bigskip

Las estafas, en concreto en las plataformas de crowdfunding abarcan muchos campos, y estas se dan cuando una campaña de crowdfunding solicita y acepta donaciones a través de falsos pretextos, engañando a las personas sobre el resultado, la naturaleza del proyecto y/o la causa solicitada.

\bigskip

También se dan casos de malversación de fondos, o el uso del proyecto para el blanqueo de capital, ya sea modificando el valor de las donaciones recibidas, o mediante una idea ilusoria de responsabilidad ecológica ( greenwashing ).

\bigskip

En este trabajo se investiga, desarrolla y propone una solución a este tipo de problemas mediante un sistema de gestión descentralizado de la financiación de proyectos.