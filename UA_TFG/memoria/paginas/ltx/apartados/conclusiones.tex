\section{Conclusiones}
\subsection{Conclusión sobre el trabajo}

Los objetivos establecidos al inicio de este trabajo se han logrado satisfactoriamente. Se ha investigado, creado e implementado un sistema de gestión descentralizado para la financiación de proyectos, el cual permite a los participantes proponer, votar, verificar y financiar proyectos de manera transparente. Este enfoque elimina la necesidad de intermediarios que puedan desviar fondos o manipular propuestas, logrando así un sistema autónomo y descentralizado en el que cualquier individuo puede presentar un proyecto. Además, tanto la financiación como el desarrollo del proyecto no se ven afectados por terceros, garantizando la ausencia de censura de cualquier tipo.


\subsection{Beneficios personales}

El desarrollo de este trabajo ha brindado una valiosa oportunidad para ampliar mis horizontes y aplicar los conocimientos adquiridos a lo largo de los años previos y durante mi carrera en un proyecto concreto. Asimismo, me ha permitido aprender y perfeccionar mis habilidades en la elaboración de documentos de esta índole, así como en la investigación y análisis de artículos científicos.

\bigskip

Gracias a este proyecto, he aprendido en detalle como funciona la tecnología blockchain, y como aplicarla en un proyecto real, además de aplicar varios patrones de desarrollo de software aprendidos durante la carrera.

\subsection{Sobre la tecnología}

Originalmente, la tecnología blockchain emergió como una propuesta alternativa al sistema económico convencional. Esta propuesta se centraba en el uso de la criptografía como medio para conseguir una plataforma económica inmune a la censura y la manipulación centralizada por parte de las empresas y los gobiernos. Con el transcurso del tiempo, esta idea inicial experimentó una evolución significativa, se empezaron a desarrollar más tecnologías que aprovechaban el potencial de la arquitectura descentralizada y la criptografía en la que se fundamenta la blockchain. En este contexto es donde surge Ethereum, un protocolo que permite la ejecución de código y el almacenamiento de datos en una blockchain descentralizada.

\bigskip

La implementación de esta tecnología posibilita el desarrollo de sistemas caracterizados por su inmutabilidad, donde además todos los procesos son trazables y los datos transparentes. Esto permite la creación de aplicaciones en las que los usuarios pueden validar los procesos en vez de confiar ciegamente en que el sistema funciona como se afirma. De este modo, se incrementa la confiabilidad y la seguridad, ya que los usuarios pueden verificar por sí mismos las operaciones y los datos.

\bigskip

El enfoque propuesto en este estudio podría haberse llevado a cabo utilizando una opción centralizada, incorporando una arquitectura tradicional cliente-servidor. Sin embargo, un sistema de tal naturaleza estaría expuesto a manipulaciones externas de diversas formas, incluso si se diseñara para operar de manera intrínsecamente autónoma.

\bigskip

Por consiguiente, la plataforma que se diseña en este trabajo de un sistema de gestión descentralizado, se beneficia directamente de las ventajas proporcionadas por la tecnología blockchain. Esta tecnología, por su propia naturaleza, asegura la transparencia y la trazabilidad tanto de los procesos como de los datos. En otras palabras, los procesos y los datos almacenados en una blockchain pueden ser rastreados y verificados por todos los participantes, lo que refuerza la integridad del sistema.

\subsection{Trabajo a futuro}

La solución propuesta y el proyecto presentan un considerable margen para futuras mejoras. Hay posibilidad de expansión mediante la adición de nuevas funcionalidades y mejoras en el diseño. Tal como se ha señalado en el capitulo \textit{Modelo del Sistema}, el proyecto enfrenta un dilema legal en relación al tratamiento de los datos personales y legales de los proponentes y los peritos. Este aspecto, por ejemplo, representa una valiosa oportunidad para su desarrollo y mejora en un futuro.
