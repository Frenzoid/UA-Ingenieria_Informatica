\subsection{Herramientas empleadas para la elaboración de este documento}

Para la realización de esta memoria se han utilizado las siguientes herramientas:

\begin{itemize}
    \item \textcolor{blue}{\href{https://en.wikipedia.org/wiki/TeX}{\textbf{Tex}}}\footnote{\url{https://en.wikipedia.org/wiki/TeX}}: Sistema software para la preparación de documentos.
    
    \item \textcolor{blue}{\href{https://app.diagrams.net/}{\textbf{Draw.io}}}\footnote{\url{https://app.diagrams.net/}}: Herramienta web para la creación y diseño de diagramas y dibujos.

    \item \textcolor{blue}{\href{https://es.overleaf.com/}{\textbf{Overleaf}}}\footnote{\url{https://es.overleaf.com}}: Herramienta web para el  desarrollo cómodo de documentos en LaTeX.
\end{itemize}

\subsection{Objetivos}

El objetivo de este trabajo es el desarrollo conceptual y el de una prueba de concepto de un sistema descentralizado para la gestión de la financiación de proyectos, similar al funcionamiento de las plataformas crowdfunding\cite{crowdfunding} actuales, pero solventando varios de los mayores problemas a los que se enfrenta este tipo de plataformas, la detección de proyectos fraudulentos y la aplicación de contra-medidas.


\subsubsection{Objetivos Específicos}
Para determinar mejor el objetivo del trabajo, este se ha dividido en los siguientes puntos:

\begin{itemize}

    \item Estado del arte de las plataformas de financiamiento colectivo de proyectos centralizadas y descentralizadas: comparativa entre ambas, ventajas y desventajas de cada tipo, desafíos a los que se enfrentan estas plataformas y reflexión acerca de la aplicación de la tecnología Blockchain en este ámbito.

    \item Análisis y Investigación de las tecnologías blockchain para aprovechar la trazabilidad y inmutabilidad que aporta esta tecnología. En concreto analizar el Protocolo Ethereum sobre como funciona en detalle, y como se puede desarrollar una aplicación usando esta blockchain.

    \item Diseñar un sistema prototipo reducido de de una Organización Autónoma Descentralizada en la Blockchain que implemente las funcionalidades necesarias orientadas a una solución a los problemas actuales los cuales afrontan estas plataformas, y que servirá como \texit{Back-End}\footnote{Back-End: Sistema software que soporta la funcionalidad de una aplicación y no es visible para el usuario final.}.


    \item Desarrollo de un \texit{Front-End}\footnote{Front-End: Interfaz de usuario basada en una aplicación web, se encarga de la interacción con el Back-End y la presentación visual de los datos.} que permita interactuar con la DAO a través de una página web, usando tecnologías de vanguardia basadas en JavaScript.



\end{itemize}