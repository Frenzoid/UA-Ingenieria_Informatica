\section{Introducción}
\subsection{Motivación}
La motivación de este trabajo viene de la pasión por el aprendizaje de la tecnología blockchain, una tecnología emergente y disruptiva, capaz de proveer de características novedosas a Internet.

\bigskip

En concreto, todo empezó en el verano del 2020, cuándo BitCoin hizo su boom económico. El entusiasmo provocado por este evento hizo que muchos llegarán a conocer esta tecnología que hasta ahora no tenia tanta fama, entre ellos yo ( Elvi ). Tuve curiosidad con respecto a la tecnología blockchain, concretamente el protocolo Ethereum, y dediqué el verano a estudiar sus utilidades y la metodología de programación empleada para crear aplicaciones sobre estos protocolos.

\bigskip

Así, durante el verano aprendí a programar usando estos protocolos, llegué a desarrollar pequeños prototipos y criptomonedas, pero nada realmente digno de mencionar. Un año más tarde, a principios del 2022, Higinio Mora que en ese momento fue mi profesor de la asignatura de Sistemas Embebidos publicó varias convocatorias de  prácticas extracurriculares en la Universidad, curiosamente relacionadas con la investigación de estas nuevas tecnologías.

\bigskip

Gracias a estas prácticas, tuve la oportunidad de poner en uso lo aprendido por mi cuenta, además de instruirme en otros protocolos blockchain como Hyperledger Fabric y IoTa. Durante estas prácticas desarrolle varios prototipos en la red Polygon, tuve el impulso para cursar y graduarme en la plataforma de \href{learnweb3.io/}{LW3}\footnote{\url{learnweb3.io}} así formalizando lo aprendido durante todo este tiempo, y además publicar un articulo\cite{articulo} en el congreso de RiiForum 2022, desarrollando una solución en relación con el problema de la gestión de la identidad digital empresarial en Internet basada en tecnología blockchain.

\bigskip

Gracias a estas experiencias, y a la pasión por nuevas tecnologías, he decido enfocar mi trabajo fin de grado a la investigación de un sistema de gestión descentralizado de la financiación de proyectos basado en Blockchain, ya que esta tecnología es adecuada para la gestión de activos de forma electrónica.


\subsection{Sobre este trabajo}

Para solventar el problema descrito en el preámbulo, el objetivo de este trabajo y solución se que propone en este documento es el uso de la tecnología blockchain para proveer de trazabilidad, confianza y inmutabilidad a los activos y eventos que se realizan durante el proceso de desarrollo de dichos proyectos.

\bigskip

Para ello, en este trabajo se estudiará las tecnologías necesarias para desarrollar un sistema descentralizado, y la integración de esta con una blockchain, además de detallar las tecnologías necesarias para desplegar y desarrollar un sistema capaz de gestionar un activo en la blockchain, y el desarrollo de la propia.