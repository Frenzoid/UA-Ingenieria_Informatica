\section{Aspectos Éticos, Económicos y Sociales}

En esta apartado, se profundiza en ciertos aspectos así destacando los aspectos éticos, económicos y sociales del proyecto.

\bigskip

Estas aspectos son clave para destacar diferentes puntos de vista, sobre el proyecto y sobre la tecnología blockchain ya que difiere radicalmente de un sistema común, y así poder debatir con un contexto más solido la naturaleza del trabajo.

\subsection{Aspectos Éticos}

\begin{itemize}
    \item \textbf{Transparencia}: La tecnología blockchain garantiza la trazabilidad y transparencia en el uso de los fondos, permitiendo a los usuarios verificar en qué se gastan los activos y asegurando un control adecuado de los recursos.
    
    \item \textbf{Participación comunitaria}: La comunidad tiene la posibilidad de votar, comentar y validar los proyectos propuestos y sus hitos, fomentando la colaboración y el compromiso ético en la toma de decisiones.
    
    \item \textbf{Protección contra malversación de fondos}: El mecanismo de gestión autónomo y la blockchain garantizan que ningún administrador o entidad pueda malversar los fondos de la plataforma.
\end{itemize}

\subsection{Aspectos Económicos}

\begin{itemize}
    \item \textbf{Financiación descentralizada}: El sistema de gestión propuesto permitiría la financiación descentralizada de proyectos, eliminando intermediarios y reduciendo los costos asociados a la gestión de fondos.
    
    \item \textbf{Inversión comunitaria}: Los usuarios pueden invertir en proyectos aprobados, diversificando sus inversiones y apoyando proyectos en los que creen.
    
    \item \textbf{Reembolsos a inversores}: En caso de que un proyecto no alcance sus metas, los fondos serán reembolsados a los inversores, protegiendo así sus intereses.
\end{itemize}

\subsection{Aspectos Sociales}

\begin{itemize}
    \item \textbf{Empoderamiento de la comunidad}: El sistema permite a la comunidad decidir qué proyectos se aprobarán, promoviendo la toma de decisiones colectiva y el empoderamiento de sus miembros.
    
    \item \textbf{Interacción y colaboración}: La posibilidad de comentar y debatir sobre los proyectos propuestos fomenta la interacción entre los usuarios y fortalece el sentido de comunidad.
    
    \item \textbf{Estímulo a la innovación}: Al facilitar la financiación y el apoyo a proyectos que de otro modo podrían tener dificultades para acceder a recursos, se promueve la innovación y el desarrollo de nuevas ideas y soluciones.
\end{itemize}